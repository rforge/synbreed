\documentclass[12pt, a4paper, titlepage, ngerman]{book}

% Lade wichtige Packete

%%% Standard-Pakete
\usepackage[latin1]{inputenc} % Dateien sind in latin1 (ISO8859-1)-Zeichenkodierung
\usepackage[T1]{fontenc}      % T1-Kodierung f�r west- und osteurop�ischen Sprachen mit lateinischem Alphabet
\usepackage[ngerman]{babel}   % Sprache ist deutsch
\usepackage{graphicx}

%%% Weitere Pakete
\usepackage{Sweave}     % zur Verwendung von Sweave
\usepackage{lmodern}  % Ersatz f�r Computer Modern-Schriften, alternativ auch Palatino (oder mathpazo, falls im Text mathematische Symbole vorkommen)
\usepackage{colortbl}
\usepackage{xcolor}
\usepackage{amsmath}
\usepackage{amstext}
\usepackage{amsfonts}
\usepackage{amssymb}
\usepackage{subfig}
\usepackage{multirow} 
\usepackage{pdfpages} % zur Einbindung von pdf's mit \includepdf()
\usepackage{setspace} % Regelt den Zeilenabstand
\usepackage[automark, headsepline]{scrpage2} % Steuert Kopf- und Fu�zeile
\usepackage[round]{natbib}   % for enhanced citing features
% \usepackage{bbm}      % for double stem math symbols like \mathbbm{1}
\usepackage{float}    % for float specifier [H]
\usepackage{picins}   % for text floating figures
\usepackage[inner=3cm, outer=3cm, top=3cm, bottom=3cm]{geometry} % regelt das Seitenlayout
\usepackage{algorithm}

%%% Stil
\onehalfspacing
%\setkomafont{caption}{\small}
\setlength{\skip\footins}{3ex}
\setlength{\parskip}{1ex}
\setlength{\parindent}{0cm}
\setlength{\headheight}{1.1\baselineskip}
\clearscrheadfoot
\ihead[]{\headmark}
\ohead[]{\pagemark}
\pagestyle{plain}


%%% Eigene Befehle
\newcommand{\bs}[1]{\boldsymbol{#1}}
\newcommand{\pdfODERpng}{png}   % Schalter: F�r h�chste Qualit�t pdf statt png verwenden
\newcommand{\geg}{\,\vert\,}
\newcommand{\abs}[1]{\lvert#1\rvert}
\newcommand{\norm}[1]{\lVert#1\rVert}
\newcommand{\IR}{\mathbbm{R}}
\newcommand{\IN}{\mathbbm{N}}
\newcommand{\IZ}{\mathbbm{Z}}
\newcommand{\id}{\mathbbm{1}}
\newcommand{\E}{\operatorname{E}}     % Erwartungswert
\newcommand{\Lik}{\operatorname{L}}     % Likelihood
\newcommand{\Var}{\operatorname{Var}} % Varianz
\newcommand{\Cov}{\operatorname{Cov}} % Kovarianz
\newcommand{\N}{\operatorname{N}}     % Normalverteilung
\newcommand{\Po}{\operatorname{Po}}   % Poissonverteilung
\newcommand{\rg}{\operatorname{rg}}   % Rang
\newcommand{\diag}{\operatorname{diag}}   % Diagonalmatrix
\newcommand{\Spur}{\operatorname{Spur}}   % Spur einer Matrix


\newcommand{\m}{\mu_{1\mid jk}}
\newcommand{\vtheta}{\boldmath$\theta$\unboldmath\,}
\newcommand{\vgamma}{\boldmath$\gamma$\unboldmath\,}
\newcommand{\vtau}{\boldmath$\tau$\unboldmath\,}
\newcommand{\vbeta}{\boldmath$\beta$\unboldmath\,}
\newcommand{\vq}{\boldmath$q$\unboldmath\,}
\newcommand{\mvtheta}{\mbox{\boldmath$\theta$\unboldmath}}
\newcommand{\mvgamma}{\mbox{\boldmath$\gamma$\unboldmath}}
\newcommand{\mvtau}{\mbox{\boldmath$\tau$\unboldmath}}
\newcommand{\mvbeta}{\mbox{\boldmath$\beta$\unboldmath}}
\newcommand{\mvq}{\mbox{\boldmath$q$\unboldmath}}
\newcommand{\ml}{{\scriptscriptstyle ML}}
\newcommand{\pml}{{\scriptscriptstyle PML}}
\newcommand{\fll}{{\scriptscriptstyle FILL}}
\newcommand{\bfepsilon}{\mbox{\boldmath$ \epsilon \, $\unboldmath } }
\newcommand{\bfbeta}{\mbox{\boldmath$ \beta $\unboldmath } \,}
\newcommand{\bfM}{\mbox{\boldmath$ M $\unboldmath } \,}
\newcommand{\bfQ}{\mbox{\boldmath$ Q $\unboldmath } \,}
\newcommand{\bfmu}{\mbox{\boldmath$ \mu $\unboldmath} \,}
\newcommand{\bftheta}{\mbox{\boldmath$ \theta $\unboldmath} \,}
\newcommand{\bfalpha}{\mbox{\boldmath$ \alpha $\unboldmath} \,}
\newcommand{\bfomega}{\mbox{\boldmath$ \omega $\unboldmath} \,}
\newcommand{\bfrho}{\mbox{\boldmath$ \rho $\unboldmath} \,}
\newcommand{\bfr}{\mbox{\boldmath$ r $\unboldmath} \,}
\newcommand{\bfpi}{\mbox{\boldmath$ \pi $\unboldmath} \,}
\newcommand{\bfSigma}{\mbox{\boldmath$ \Sigma $\unboldmath} \,}
\newcommand{\bfgamma}{\mbox{\boldmath$ \gamma $\unboldmath} \,}
\newcommand{\bfDelta}{\mbox{\boldmath$ \Delta $\unboldmath} \,}
\newcommand{\bflambda}{\mbox{\boldmath$ \lambda $\unboldmath} \,}
\newcommand{\bfeta}{\mbox{\boldmath$ \eta $\unboldmath} \,}
\newcommand{\bfdelta}{\mbox{\boldmath$ \delta $\unboldmath} \,}

\newtheorem{bemerkung}{Bemerkung}[chapter]
\newtheorem{defn}{Definition}[chapter]
\newtheorem{satz}{Satz}[chapter]
\newenvironment{sat}{\begin{satz}\rm
                       }{\end{satz}}
\newenvironment{example}{\vs
           \refstepcounter{beisp} \mbox{{\bf
           Beispiel \arabic{chapter}.\arabic{beisp}:}}  \rm \small
                      }{\normalsize
                        \vs}
\newenvironment{define}{\begin{defn}\rm
                       }{\end{defn}}

%%% hyperref-Paket, PDF-Eigenschaften
\usepackage[pdftex]{hyperref}
\hypersetup{
    pdftitle={Statistische Analysen zu Training und Leistung im Skisprung},
    pdfauthor={Valentin Wimmer},
    pdfsubject={Masterarbeit},
    pdfkeywords={Skisprung, Leistungsdiagnostik, Multivariate Verfahren, Faktorenanalyse},
    colorlinks=true,
    linkcolor=black,
    urlcolor=blue,
    citecolor=black
}

%%% Metadaten
\title{Vorlesungsskript zu Statistische Methoden\\[1em]
{\Large Technische Universit�t M�nchen\\Wissenschaftzentrum Weihenstephan\\
Lehrstuhl f�r Pflanzenz�chtung}}
\author{ \Large Prof. Dr. Chris-Carolin Sch�n und Valentin Wimmer, M.Sc}
\date{ \Large \today}



\begin{document}

%--- TITELSEITE ---

\pagenumbering{roman}
\maketitle

%--- VORWORT ---

\include{Vorwort}
\cleardoublepage


%--- INHALTSVERZEICHNIS ---

\tableofcontents
\cleardoublepage

%--- HAUPTTEIL ---

% Format der Seitenzahlen
\pagenumbering{arabic}
\pagestyle{scrheadings}

\include{Einf�hrung}
\include{StatistischeMasszahlen}
\include{Anova}
\include{Regression}
\chapter{Matrixalgebra}\label{ch:Matrixalgebra}


\section{Definitionen}

\begin{defn}
Eine $m \times n$-Matrix ${\bf A}$ ist eine rechteckige Anordnung von Elementen
(in diesem Buch und Anhang : reelle Zahlen) in $m$ Zeilen und $n$ Spalten.
\end{defn}

Wir sagen, ${\bf A}$ sei vom Typ $m\times n$ oder $(m,n)$ und
schreiben h�ufig zur Abk�rzung $\begin{array}[t]{c} {\bf A} \\
{\scriptstyle m,n} \end{array}$, $\begin{array}[t]{c} {\bf A} \\
{\scriptscriptstyle m \times n} \end{array}$ oder ${\bf A}:(m,n)$ .

Sei $a_{ij}$ das Element in der i-ten Zeile und der j-ten Spalte von ${\bf A}$. Dann
ist
$$
{\bf A} = \left( \begin{array}{cccc} a_{11} & a_{12} & \cdots & a_{1n} \\
                               a_{21} & a_{22} & \cdots & a_{2n} \\
                               \vdots & \vdots &        & \cdots \\
                               a_{m1} & a_{m2} & \cdots & a_{mn}
    \end{array} \right) = (a_{ij}) .
$$
Eine Matrix mit $n=m$ Zeilen und Spalten hei�t quadratisch.

Eine quadratische Matrix mit Nullen unterhalb der Diagonalen hei�t
obere Dreiecksmatrix.

\begin{defn}

Die Transponierte ${\bf A}'$ einer Matrix ${\bf A}$ entsteht aus ${\bf A}$ durch Vertauschen von
Zeilen und Spalten.Damit ist
$$
\begin{array}[t]{c} {\bf A}' \\ n,m \end{array} = (a_{ji}) .
$$
\end{defn}

Es gilt
$$
({\bf A}')'={\bf A},\quad ({\bf A}+{\bf B})' = {\bf A}' + {\bf B}',
\quad ({\bf AB})'={\bf B}'{\bf A}.'
$$

\begin{defn}

Eine quadratische Matrix hei�t symmetrisch, falls ${\bf A}'={\bf
A}$.
\end{defn}

\begin{defn}

Eine $m \times 1$-Matrix ${\bf A}$ hei�t Spaltenvektor ${\bf a}$,
d.h.
$$
{\bf a} = \left( \begin{array}{c} a_1 \\ \vdots \\ a_m \end{array} \right) .
$$
\end{defn}

\begin{defn}

Eine $1 \times n$-Matrix ${\bf A}$ hei�t Zeilenvektor ${\bf a}'$,
d.h.
$$
{\bf a}' = (a_1, \cdots, a_n) .
$$
\end{defn}

Damit existieren f�r eine Matrix ${\bf A}$ folgende alternative
Darstellungen
$$
\begin{array}[t]{c} {\bf A} \\ m,n \end{array}
= (\begin{array}[t]{r} {\bf a}_{(1)} \\ m,1 \end{array}
, \cdots , \begin{array}[t]{r}{\bf a}_{(n)} \\ m,1 \end{array} ) =
\left( \begin{array}{c} {\bf a}_1' \\ \vdots \\ {\bf a}_m' \end{array} \right)
\begin{array}{c} (1,n) \\ \vdots \\ (1,n) \end{array}
$$
mit
$$
{\bf a}_{(j)} = \left( \begin{array}{c} a_{1j} \\ \vdots \\ a_{mj} \end{array}
\right), \quad {\bf a}_i = \left( \begin{array}{c} a_{i1} \\ \vdots \\ a_{in}
\end{array} \right) .
$$

\begin{defn}

Der $1 \times n$ Vektor $(1, \cdots, 1)$ wird mit ${\bf 1}_n'$ oder kurz
${\bf 1}'$ bezeichnet.

\end{defn}

\begin{defn}

Die $(m,m)$-Matrix ${\bf A}$ mit $a_{ij} = 1$ (alle i,j) wird mit
$$
{\bf J}_{m} = \left(\begin{array}{ccc} 1 & \cdots & 1 \\
                             \vdots & & \vdots \\
                             1 & \vdots & 1
    \end{array} \right) = {\bf 1}_m {\bf 1}_m'
$$
bezeichnet.
\end{defn}

\begin{defn}

Der $1 \times n$-Zeilenvektor
$$
{\bf e}_i' = \begin{array}[t]{ccc}(0, \cdots,0,&1&, 0, 0, \cdots, 0)\\
                               &i&
\end{array}
$$
mit einer $1$ an der i-ten Stelle hei�t $i$--ter Einheitsvektor.
\end{defn}

\begin{defn}

Die quadratische $(n,n)$-Matrix mit Einsen auf der Hauptdiagonalen
und Nullen sonst hei�t Einheitsmatrix ${\bf I}_n$.
\end{defn}

\begin{defn}

Eine quadratische Matrix ${\bf A}$ mit Elementen $a_{ii}$ auf der
Hauptdiagonalen und Nullen sonst hei�t Diagonalmatrix. Wir schreiben
$$
\begin{array}[t]{c}{\bf A} \\ {\scriptstyle n,n} \end{array} = \mbox{diag}
(a_{11}, \cdots, a_{nn}) = \mbox{diag} (a_{ii}) =
\left( \begin{array}{ccc} a_{11} & & 0 \\
                             & \ddots & \\
                          0 & & a_{nn}
\end{array} \right).
$$
\end{defn}

\begin{defn}

Eine Matrix ${\bf A}$, die als Zusammenfassung von Submatrizen
dargestellt wird, hei�t unterteilt oder
partitioniert.\index{partioniert}
\end{defn}

Beispiele sind
\begin{eqnarray*}
\begin{array}[t]{c} {\bf A} \\ {\scriptstyle m,n} \end{array} &=&
( \begin{array}[t]{c} {\bf A}_1 \\ {\scriptstyle m,r} \end{array} ,
  \begin{array}[t]{c} {\bf A}_2 \\ {\scriptstyle m,s} \end{array} )
\quad \mbox{mit} \: r+s = n \\
\begin{array}[t]{c} {\bf A} \\ {\scriptstyle m,n} \end{array} &=&
\left( \begin{array}{cc} {\bf A}_{11} & {\bf A}_{12} \\
{\bf A}_{21} & {\bf A}_{22} \end{array}
\right)
\end{eqnarray*}
mit den Dimensionen der Submatrizen $ \left(\begin{array}{cc} r,r & r,s \\
m-r,r & m-r,s \end{array} \right) $.

F�r partitionierte Matrizen gilt z.B.
$$
{\bf A}' = \left( \begin{array}{cc}  {\bf A}'_{11} &
{\bf A}'_{21} \\ {\bf A}'_{12} & {\bf A}'_{22}
\end{array} \right), \quad {\bf A}' = \left( \begin{array}{c} {\bf A}_1' \\
{\bf A}_2'
\end{array} \right) .
$$

\section{Rechnenregeln}

\begin{frame}[fragile]\frametitle{Codieren der genotypischen Daten}
\begin{algorithm}[H]
\caption{Umkodieren der Allele}
\begin{algorithmic}[1]
\FOR{$j=1$ to $M$}
\STATE Erstelle H�ufigkeitstabelle der Allele $a_1$,...,$a_{n(j)}$ f�r Marker $j$
\STATE Sortiere absteigend nach H�ufigkeit $f(a_{(1)}) > f(a_{(1)}) > ...$
\FOR{$i=1$ to $n$}
\IF{$g_{ij}$ = $a_{(1)}$}
\STATE  $g_{ij}$ = 1
\IF{$g_{ij}$ = $a_{(2)}$}
\STATE  $g_{ij}$ = 0
\ELSE{}
\STATE  $g_{ij}$ = NA
\ENDFOR
\ENDFOR
\end{algorithmic}
\end{algorithm}
\include{MultipleRegression}
\include{Versuchsdesigns}
\include{MultiplesTesten}
\include{GemischteModelle}
\include{GeneralisierteRegression}

\bibliographystyle{natdin}
\bibliography{Master}

\end{document}
