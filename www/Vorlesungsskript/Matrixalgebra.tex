\chapter{Matrixalgebra}\label{ch:Matrixalgebra}


\section{Definitionen}

\begin{defn}
Eine $m \times n$-Matrix ${\bf A}$ ist eine rechteckige Anordnung von Elementen
(in diesem Buch und Anhang : reelle Zahlen) in $m$ Zeilen und $n$ Spalten.
\end{defn}

Wir sagen, ${\bf A}$ sei vom Typ $m\times n$ oder $(m,n)$ und
schreiben h�ufig zur Abk�rzung $\begin{array}[t]{c} {\bf A} \\
{\scriptstyle m,n} \end{array}$, $\begin{array}[t]{c} {\bf A} \\
{\scriptscriptstyle m \times n} \end{array}$ oder ${\bf A}:(m,n)$ .

Sei $a_{ij}$ das Element in der i-ten Zeile und der j-ten Spalte von ${\bf A}$. Dann
ist
$$
{\bf A} = \left( \begin{array}{cccc} a_{11} & a_{12} & \cdots & a_{1n} \\
                               a_{21} & a_{22} & \cdots & a_{2n} \\
                               \vdots & \vdots &        & \cdots \\
                               a_{m1} & a_{m2} & \cdots & a_{mn}
    \end{array} \right) = (a_{ij}) .
$$
Eine Matrix mit $n=m$ Zeilen und Spalten hei�t quadratisch.

Eine quadratische Matrix mit Nullen unterhalb der Diagonalen hei�t
obere Dreiecksmatrix.

\begin{defn}

Die Transponierte ${\bf A}'$ einer Matrix ${\bf A}$ entsteht aus ${\bf A}$ durch Vertauschen von
Zeilen und Spalten.Damit ist
$$
\begin{array}[t]{c} {\bf A}' \\ n,m \end{array} = (a_{ji}) .
$$
\end{defn}

Es gilt
$$
({\bf A}')'={\bf A},\quad ({\bf A}+{\bf B})' = {\bf A}' + {\bf B}',
\quad ({\bf AB})'={\bf B}'{\bf A}.'
$$

\begin{defn}

Eine quadratische Matrix hei�t symmetrisch, falls ${\bf A}'={\bf
A}$.
\end{defn}

\begin{defn}

Eine $m \times 1$-Matrix ${\bf A}$ hei�t Spaltenvektor ${\bf a}$,
d.h.
$$
{\bf a} = \left( \begin{array}{c} a_1 \\ \vdots \\ a_m \end{array} \right) .
$$
\end{defn}

\begin{defn}

Eine $1 \times n$-Matrix ${\bf A}$ hei�t Zeilenvektor ${\bf a}'$,
d.h.
$$
{\bf a}' = (a_1, \cdots, a_n) .
$$
\end{defn}

Damit existieren f�r eine Matrix ${\bf A}$ folgende alternative
Darstellungen
$$
\begin{array}[t]{c} {\bf A} \\ m,n \end{array}
= (\begin{array}[t]{r} {\bf a}_{(1)} \\ m,1 \end{array}
, \cdots , \begin{array}[t]{r}{\bf a}_{(n)} \\ m,1 \end{array} ) =
\left( \begin{array}{c} {\bf a}_1' \\ \vdots \\ {\bf a}_m' \end{array} \right)
\begin{array}{c} (1,n) \\ \vdots \\ (1,n) \end{array}
$$
mit
$$
{\bf a}_{(j)} = \left( \begin{array}{c} a_{1j} \\ \vdots \\ a_{mj} \end{array}
\right), \quad {\bf a}_i = \left( \begin{array}{c} a_{i1} \\ \vdots \\ a_{in}
\end{array} \right) .
$$

\begin{defn}

Der $1 \times n$ Vektor $(1, \cdots, 1)$ wird mit ${\bf 1}_n'$ oder kurz
${\bf 1}'$ bezeichnet.

\end{defn}

\begin{defn}

Die $(m,m)$-Matrix ${\bf A}$ mit $a_{ij} = 1$ (alle i,j) wird mit
$$
{\bf J}_{m} = \left(\begin{array}{ccc} 1 & \cdots & 1 \\
                             \vdots & & \vdots \\
                             1 & \vdots & 1
    \end{array} \right) = {\bf 1}_m {\bf 1}_m'
$$
bezeichnet.
\end{defn}

\begin{defn}

Der $1 \times n$-Zeilenvektor
$$
{\bf e}_i' = \begin{array}[t]{ccc}(0, \cdots,0,&1&, 0, 0, \cdots, 0)\\
                               &i&
\end{array}
$$
mit einer $1$ an der i-ten Stelle hei�t $i$--ter Einheitsvektor.
\end{defn}

\begin{defn}

Die quadratische $(n,n)$-Matrix mit Einsen auf der Hauptdiagonalen
und Nullen sonst hei�t Einheitsmatrix ${\bf I}_n$.
\end{defn}

\begin{defn}

Eine quadratische Matrix ${\bf A}$ mit Elementen $a_{ii}$ auf der
Hauptdiagonalen und Nullen sonst hei�t Diagonalmatrix. Wir schreiben
$$
\begin{array}[t]{c}{\bf A} \\ {\scriptstyle n,n} \end{array} = \mbox{diag}
(a_{11}, \cdots, a_{nn}) = \mbox{diag} (a_{ii}) =
\left( \begin{array}{ccc} a_{11} & & 0 \\
                             & \ddots & \\
                          0 & & a_{nn}
\end{array} \right).
$$
\end{defn}

\begin{defn}

Eine Matrix ${\bf A}$, die als Zusammenfassung von Submatrizen
dargestellt wird, hei�t unterteilt oder
partitioniert.\index{partioniert}
\end{defn}

Beispiele sind
\begin{eqnarray*}
\begin{array}[t]{c} {\bf A} \\ {\scriptstyle m,n} \end{array} &=&
( \begin{array}[t]{c} {\bf A}_1 \\ {\scriptstyle m,r} \end{array} ,
  \begin{array}[t]{c} {\bf A}_2 \\ {\scriptstyle m,s} \end{array} )
\quad \mbox{mit} \: r+s = n \\
\begin{array}[t]{c} {\bf A} \\ {\scriptstyle m,n} \end{array} &=&
\left( \begin{array}{cc} {\bf A}_{11} & {\bf A}_{12} \\
{\bf A}_{21} & {\bf A}_{22} \end{array}
\right)
\end{eqnarray*}
mit den Dimensionen der Submatrizen $ \left(\begin{array}{cc} r,r & r,s \\
m-r,r & m-r,s \end{array} \right) $.

F�r partitionierte Matrizen gilt z.B.
$$
{\bf A}' = \left( \begin{array}{cc}  {\bf A}'_{11} &
{\bf A}'_{21} \\ {\bf A}'_{12} & {\bf A}'_{22}
\end{array} \right), \quad {\bf A}' = \left( \begin{array}{c} {\bf A}_1' \\
{\bf A}_2'
\end{array} \right) .
$$

\section{Rechnenregeln}

\begin{frame}[fragile]\frametitle{Codieren der genotypischen Daten}
\begin{algorithm}[H]
\caption{Umkodieren der Allele}
\begin{algorithmic}[1]
\FOR{$j=1$ to $M$}
\STATE Erstelle H�ufigkeitstabelle der Allele $a_1$,...,$a_{n(j)}$ f�r Marker $j$
\STATE Sortiere absteigend nach H�ufigkeit $f(a_{(1)}) > f(a_{(1)}) > ...$
\FOR{$i=1$ to $n$}
\IF{$g_{ij}$ = $a_{(1)}$}
\STATE  $g_{ij}$ = 1
\IF{$g_{ij}$ = $a_{(2)}$}
\STATE  $g_{ij}$ = 0
\ELSE{}
\STATE  $g_{ij}$ = NA
\ENDFOR
\ENDFOR
\end{algorithmic}
\end{algorithm}